
\chapter{Surveys}
\label{ch:surveys}
\chapterauthor*{Guillaume Chéreau}

\section{Introduction}
\label{sec:surveys:introduction}


A \indexterm{sky survey} \newFeature{v0.18.0} is a map of the sky stored as a hierarchical set of a potentially
large number of smaller images (called tiles).  The advantage compared to a
regular texture is that we need to render only the visible tiles of a potentially gigantic image at the
lowest resolution needed.  This is particularly interesting for rendering
online images that can be stored on a server, while the client only has
to download the parts he currently uses.

Since version 0.18.0, Stellarium added some preliminary support for loading and
rendering online surveys in the \indexterm{Hierarchical Progressive Surveys} (HiPS) format,
developed by the \indexterm{International Virtual Observatory Alliance}.
A full description of the format can be found on the IVOA website\footnote{%
\url{http://www.ivoa.net/documents/HiPS/20170519/REC-HIPS-1.0-20170519.pdf}}.

\section{Hipslist file and default surveys}
\label{sec:surveys:hipslistFile}

Hipslist files are text files used to describe catalogs of HiPS surveys.  The
full specification is part of the HiPS format, and looks like that:

\begin{configfileScr}
# Example of a hipslist file.
# Date: 2018-03-19

hips_service_url  = https://data.stellarium.org/surveys/callisto
hips_release_date = 2018-03-18T14:01Z
hips_status       = public mirror clonableOnce
\end{configfileScr}

As of v0.18.0, by default Stellarium tries to load HiPS from two sources:
\url{http://alaskybis.unistra.fr/hipslist} and
\url{https://data.stellarium.org/surveys/hipslist}.
This can be changed with the hips/sources options in the configuration
file (see also section~\ref{sec:config.ini:hips}).  For example:

\begin{configfile}
[hips]
sources/size = 1
sources/1/url = "https://myownhips/hipslist"

\end{configfile}

\section{Solar system HiPS survey}

Though not specified in the HiPS standard, Stellarium recognises HiPS surveys
representing planet textures, as opposed to sky surveys.  If the
\texttt{obs\_frame} property of a survey is set to the name of a planet or
other solar system body, Stellarium will render it in place of the default
texture used for the body.
