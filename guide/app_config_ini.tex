%% Part of Stellarium User Guide
%% Status:
%% 2015-12-30 taken from wiki
%% 2016-04-06 GZ reformatted a bit. 
%% 2016-08-01 AW added color boxes for visual representation of colours. 
%% 2017-12-23..26 GZ removed many \midrules, and reduced labels (where not referenced) to allow processing with tex4ht.
%% TODO: document the more obscure entries. Add links to the respective chapters.
%% TODO: remove unused entries, add new ones.  


\chapter{Configuration Files}
\label{app:ConfigFiles}

\section{Program Configuration}
\label{sec:config.ini}
First, see \ref{sec:ConfigurationFile} (The
Main Configuration File) for information about the file \file{config.ini}, including its
default installed location, and command line options that can
affect how it is processed.

%Deprecated parameters are marked by gray background. 
%Possible new parameters are marked by yellow background.
%% GZ This is no longer true. Reactivate somehow? 

The file \file{config.ini} (or a file which you can load instead with
the \command{-\/-config <file>} option) is structured into the following
parts. In addition, plugins can add a section named like the plugin
(Exception: The Text User Interface plugin's section is named \texttt{[tui]}
for historical reasons).

\subsection{\big[astro\big]}
%\label{sec:config.ini:astro}

This section includes settings for the commonly displayed objects.

\begin{longtable}{l|l|l|p{55mm}}
\toprule
\emph{ID} & \emph{Type} & \emph{Default} & \emph{Description}\\\midrule
apparent\_magnitude\_algorithm     & string & ExpSup2013 & Set algorithm for computation of apparent magnitude of the planets. 
                                                   Possible values: \emph{ExpSup2013}, \emph{ExpSup1992}, \emph{AstrAlm1984}, \emph{Mueller1893} and \emph{Generic}.\\%\midrule
nebula\_magnitude\_limit           & float  & 8.5  & Value of limiting magnitude for the deep-sky objects.\\%\midrule
star\_magnitude\_limit             & float  & 6.5  & Value of limiting magnitude for the stars. Sometimes you don't want to display more stars when zooming in. \\%\midrule
planet\_magnitude\_limit  		   & float  & 6.5  & Value of limiting magnitude for the planets. Sometimes you don't want to display more planets when zooming in. \\%\midrule
planets\_orbits\_color\_style      & string & one\_color & Set the style of orbits color. Possible values: \emph{one\_color}, \emph{groups} and \emph{max}.\\%\midrule
flag\_nebula\_magnitude\_limit     & bool   & false & Set to \emph{true} to activate limit for showing deep-sky objects.\\%\midrule
flag\_star\_magnitude\_limit       & bool   & false & Set to \emph{true} to activate limit for showing stars\\%\midrule
flag\_planet\_magnitude\_limit     & bool   & false & Set to \emph{true} to activate limit for showing planets.\\\midrule
extinction\_mode\_below\_horizon   & string & zero & Set extinction mode for atmosphere below horizon. Possible values: \emph{zero}, \emph{mirror} and \emph{major\_planets}.\\\midrule
%
flag\_stars                        & bool   & true  & Set to \emph{false} to hide the stars on start-up\\%\midrule
flag\_star\_name                   & bool   & true  & Set to \emph{false} to hide the star labels on start-up\\%\midrule
flag\_star\_additional\_names      & bool   & true  & Set \emph{true} to use additional names (also known as) for stars.\\%\midrule
flag\_planets                      & bool   & true  & Set to \emph{false} to hide the planet labels on start-up\\%\midrule
flag\_planets\_hints               & bool   & true  & Set to \emph{false} to hide the planet hints on startup (names and circular highlights)\\%\midrule
flag\_planets\_orbits              & bool   & false & Set to \emph{true} to show the planet orbits on startup\\%\midrule
flag\_permanent\_orbits            & bool   & false & Set to \emph{true} to show the orbit of planet, when planet is out of the viewport also.\\%\midrule
flag\_planets\_pointers            & bool   & false & Set to \emph{true} to show the planet pointer markers on startup\\%\midrule
flag\_planets\_nomenclature        & bool   & false & Set to \emph{true} to show the planetary nomenclature markers and labels\\%\midrule
flag\_hide\_local\_nomenclature    & bool   & true & Set to \emph{true} to hide the planetary nomenclature items on the observer's celestial body\\\midrule
%
flag\_light\_travel\_time          & bool   & true  & Should be \emph{true} to improve accuracy in the movement of the planets by compensating 
                                           for the time it takes for light to travel. This has a slight impact on performance, 
                                           but is essential e.g.\ for Jupiter's moons.\\%\midrule
flag\_object\_trails               & bool   & false & Turns on and off drawing of object trails (which show the movement of the planets over time).\\%\midrule
flag\_isolated\_trails             & bool   & true  & Turns on and off drawing of isolated trails.\\%\midrule
number\_isolated\_trails           & int    & 1     & Number of isolated trails for latest selected Solar system bodies [1..5].\\%\midrule
flag\_isolated\_orbits             & bool   & true  & Turns on and off drawing of isolated orbits.\\\midrule
%
flag\_nebula                       & bool   & true  & Set to \emph{false} to hide the nebulae on start-up. \\%\midrule
flag\_nebula\_name                 & bool   & false & Set to \emph{true} to show the nebula labels on start-up. \\%\midrule
flag\_nebula\_display\_no\_texture & bool   & false & Set to \emph{true} to suppress displaying of nebula textures. \\%\midrule
nebula\_hints\_amount              & float  & 3.0   & Sets the amount of hints [0\ldots10]. \\%\midrule
nebula\_labels\_amount             & float  & 3.0   & Sets the amount of labels [0\ldots10].\\%\midrule
flag\_milky\_way                   & bool   & true  & Set to \emph{false} to hide the Milky Way.\\%\midrule
milky\_way\_intensity              & float  & 1.0   & Sets the relative brightness with which the milky way is drawn. Typical [1\ldots3]. \\%\midrule
flag\_zodiacal\_light              & bool   & true  & Set to \emph{false} to hide the zodiacal light\\%\midrule
zodiacal\_light\_intensity         & float  & 1.0   & Sets the relative brightness with which the zodiacal light is drawn. \\%\midrule
max\_mag\_nebula\_name             & float  & 8.0   & Sets the magnitude of the nebulae whose name is shown. \\%\midrule
flag\_nebula\_hints\_proportional  & bool   & false & Enables/disables proportional markers for deep-sky objects. \\%\midrule
flag\_surface\_brightness\_usage   & bool   & false & Enables/disables usage surface brightness for markers for deep-sky objects.\\\midrule
%
flag\_size\_limits\_usage          & bool   & false & Enables/disables usage angular size filter for markers for deep-sky objects.\\%\midrule
size\_limit\_min                   & int    & 1     & Minimal angular size of visible deep-sky objects in arcminutes.\\%\midrule
size\_limit\_max                   & int    & 600   & Maximum angular size of visible deep-sky objects in arcminutes.\\\midrule
%
flag\_use\_type\_filter            & bool   & false & Enables/disables usage of the type filters for deep-sky objects. \\\midrule
flag\_nutation  				   & bool   & true  & Enables/disables usage nutation.\\\midrule
flag\_topocentric\_coordinates	   & bool   & true  & Enables/disables usage topocentric coordinates.\\\midrule
%
flag\_dso\_designation\_usage      & bool   & false & If \emph{true}, use designations of deep-sky objects on the sky instead of their common names.\\%\midrule
flag\_dso\_outlines\_usage         & bool   & false & If \emph{true}, use outlines for big deep-sky objects on the sky instead of their hints.\\%\midrule
flag\_dso\_additional\_names       & bool   & true  & Set \emph{true} to use additional names (also known as) for deep-sky objects.\\\midrule
%
flag\_grs\_custom                  & bool   & false & Turns on and off to using custom settings for calculation of position of Great Red Spot on Jupiter.\\%\midrule
grs\_longitude                     & int    & 216   & Longitude (System II) of Great Red Spot on Jupiter in degrees.\\%\midrule
grs\_drift                         & float  & 15.0  & Annual drift of Great Red Spot on Jupiter in degrees.\\%\midrule
grs\_jd                            & float  & 2456901.5 & Initial JD for calculation of position of Great Red Spot on Jupiter.\\%\midrule
grs\_measurements\_url             & string &       & URL of recent measurements of GRS' position\footnote{Default value is \url{http://jupos.privat.t-online.de/rGrs.htm}}. \\\midrule
%
meteor\_zhr                        & int    & 10    & ZHR for sporadic meteors.\\\midrule
%
de430\_path                        & string &       & Path to JPL DE430 ephemerides file.\\%\midrule
flag\_use\_de430                   & bool   & false & Enables/disables usage JPL DE430 ephemerides (if available).\\%\midrule
de431\_path                        & string &       & Path to JPL DE431 ephemerides file.\\%\midrule
flag\_use\_de431                   & bool   & false & Enables/disables usage JPL DE431 ephemerides (if available).\\\midrule
%
flag\_toast\_survey                & bool   & false & Enables/disables Digitized Sky Survey in TOAST format.\\%\midrule
toast\_survey\_host                & string &       & Host of Digitized Sky Survey storage data.\\%\midrule
toast\_survey\_directory           & string & results & Directory name of Digitized Sky Survey storage data.\\%\midrule
toast\_survey\_levels              & int    & 11    & Count of levels for Digitized Sky Survey storage data.\\\bottomrule
\end{longtable}

\subsection{\big[astrocalc\big]}
%\label{sec:config.ini:astrocalc}

This section includes settings for the Astronomical Calculations feature.

\noindent%
\begin{longtable}{l|l|l|p{65mm}}
\toprule
\emph{ID}                   & \emph{Type} & \emph{Default} & \emph{Description}\\\midrule
flag\_ephemeris\_markers       & bool   & true  & Set to \emph{true} to show the calculated ephemeris markers for Solar system bodies.\\%\midrule
flag\_ephemeris\_dates         & bool   & false & Set to \emph{true} to show the dates of calculated ephemeris for Solar system bodies.\\%\midrule
flag\_ephemeris\_magnitudes    & bool   & false & Set to \emph{true} to show the magnitude of Solar system body for the calculated ephemeride.\\%\midrule
flag\_ephemeris\_horizontal    & bool   & false & Set to \emph{true} to use horizontal coordinates for ephemerides.\\%\midrule
ephemeris\_celestial\_body     & string & Moon  & Default celestial body for computation of ephemerides.\\%\midrule
ephemeris\_time\_step          & int    & 6     & Identificator for time step category.\\%\midrule
celestial\_magnitude\_limit    & float  & 6.0   & Value of limiting magnitude for the celestial objects in computation of their positions.\\%\midrule
celestial\_category            & int    & 200   & Identificator for default category of celestial bodies.\\%\midrule
flag\_horizontal\_coordinates  & bool   & false & Set to \emph{true} to use horizontal coordinates for compute of positions for the celestial objects.\\%\midrule
flag\_phenomena\_opposition    & bool   & false & Set to \emph{true} to compute of opposition phenomenon between the Solar system bodies.\\%\midrule
phenomena\_angular\_separation & float  & 1.0   & Allowed angular separation between celestial bodies for computation of phenomena.\\%\midrule
phenomena\_celestial\_body     & string & Venus & Default celestial body for computation of phenomena.\\%\midrule
phenomena\_celestial\_group    & int    & 0     & Identificator for group of celestial bodies for computation of phenomenae.\\%\midrule
wut\_magnitude\_limit          & float  & 10.0  & Value of limiting magnitude for calculations within What's Up Tonight feature.\\%\midrule
wut\_time\_interval            & int    & 0     & Identificator for time interval within What's Up Tonight feature computations.\\%\midrule
wut\_angular\_limit\_flag      & bool   & false & Set to \emph{true} to use the angular size filter.\\%\midrule
wut\_angular\_limit\_min       & float  & 1.0   & Minimal allowed angular size, arcminutes.\\%\midrule
wut\_angular\_limit\_max       & float  & 600.0 & Maximal allowed angular size, arcminutes.\\%\midrule
graphs\_celestial\_body        & string & Moon  & Default celestial body for draw of graphs.\\%\midrule
graphs\_first\_id              & int    & 1     & Identificator for left graph (blue line).\\%\midrule
graphs\_second\_id             & int    & 2     & Identificator for right graph (yellow line).\\\bottomrule
\end{longtable}

\subsection{\big[color\big]}
%\label{sec:config.ini:color}

This section defines the RGB colors for the various objects, lines,
grids, labels etc. Values are given in float from 0 to
1. e.g. \emph{1.0,1.0,1.0} for white, or \emph{1,0,0} for red. Leave
no whitespace between the numbers!

\begin{longtable}{l|l|p{55mm}}
\toprule
\emph{ID}	                            & \emph{Default}      & \emph{Colour of the\ldots}\\
\midrule
default\_color           				& \ccbox{0.5,0.5,0.7} & default colour.\\\midrule
daylight\_text\_color            		& \ccbox{0.0,0.0,0.0} & info text at daylight. \\\midrule
sky\_background\_color                  & \ccbox{0.0,0.0,0.0} & sky background (non-black is useful for artists only). \\\midrule
%% Ecliptic J2000 group: red tones
ecliptical\_J2000\_color 				& \ccbox{0.4,0.1,0.1} &  ecliptical grid (J2000.0 frame). \\%\midrule
ecliptic\_J2000\_color   				& \ccbox{0.7,0.2,0.2} &  ecliptic line (J2000.0 frame). \\%\midrule
equinox\_J2000\_points\_color			& \ccbox{0.7,0.2,0.2} &  equinox points (J2000.0 frame). \\%\midrule
solstice\_J2000\_points\_color			& \ccbox{0.7,0.2,0.2} &  solstice points (J2000.0 frame). \\%\midrule
ecliptic\_J2000\_poles\_color			& \ccbox{0.7,0.2,0.2} &  ecliptic poles (J2000.0 frame). \\\midrule
%% Ecliptic of date group: orange tones
ecliptical\_color        				& \ccbox{0.6,0.3,0.1} &  ecliptical grid (of date). \\%\midrule
ecliptic\_color          				& \ccbox{0.9,0.6,0.2} &  ecliptic line (of date). \\%\midrule
equinox\_points\_color					& \ccbox{0.9,0.6,0.2} &  equinox points (of date). \\%\midrule
solstice\_points\_color					& \ccbox{0.9,0.6,0.2} &  solstice points (of date). \\%\midrule
ecliptic\_poles\_color					& \ccbox{0.9,0.6,0.2} &  ecliptic poles (of date). \\%\midrule
precession\_circles\_color 				& \ccbox{0.9,0.6,0.2} &  precession circles. \\\midrule
%% Equatorial J2000 group: blue tones (standard atlas coordinates)
equator\_J2000\_color      				& \ccbox{0.2,0.2,0.6} &  equator line (J2000.0 frame). \\%\midrule
equatorial\_J2000\_color 				& \ccbox{0.1,0.1,0.5} &  equatorial grid (J2000.0 frame). \\%\midrule
celestial\_J2000\_poles\_color			& \ccbox{0.2,0.2,0.6} &  celestial poles (J2000.0 frame). \\\midrule
%% Equatorial of date group: brighter blue
equator\_color           				& \ccbox{0.3,0.5,1.0} &  equator line (of date). \\%\midrule
equatorial\_color        				& \ccbox{0.2,0.3,0.8} &  equatorial grid (of date). \\%\midrule
celestial\_poles\_color					& \ccbox{0.3,0.5,1.0} &  celestial poles (of date). \\%\midrule
circumpolar\_circles\_color 			& \ccbox{0.3,0.5,1.0} &  circumpolar circles. \\\midrule
%% Galaxy: brownish, not too strong.
galactic\_color          				& \ccbox{0.3,0.2,0.1} &  galactic grid. \\%\midrule
galactic\_equator\_color 				& \ccbox{0.5,0.3,0.1} &  galactic equator line. \\%\midrule
galactic\_poles\_color					& \ccbox{0.5,0.3,0.1} &  galactic poles. \\\midrule
%% Supergalactic group: dark grey, should be less apparent than galactic.
supergalactic\_color       				& \ccbox{0.2,0.2,0.2} &  supergalactic grid. \\%\midrule
supergalactic\_equator\_color 			& \ccbox{0.4,0.4,0.4} &  supergalactic equator line. \\%\midrule
supergalactic\_poles\_color				& \ccbox{0.4,0.4,0.4} &  supergalactic poles. \\\midrule
%% Horizon and altazimuthal grid: greenish.
azimuthal\_color         				& \ccbox{0.0,0.3,0.2} &  azimuthal grid. \\%\midrule
horizon\_color           				& \ccbox{0.2,0.6,0.2} &  horizon line. \\%\midrule
meridian\_color          				& \ccbox{0.2,0.6,0.2} &  meridian line. \\%\midrule
prime\_vertical\_color   				& \ccbox{0.2,0.5,0.2} &  prime vertical. \\%\midrule
zenith\_nadir\_color					& \ccbox{0.2,0.6,0.2} &  zenith and nadir. \\%\midrule
cardinal\_color          				& \ccbox{0.8,0.2,0.1} &  cardinal points. \\\midrule
%% A mix of equatorial (blueish) and ecliptical (reddish)...
colures\_color 			 			& \ccbox{0.5,0.0,0.5} &  colures. \\\midrule
antisolar\_point\_color 				& \ccbox{0.9,0.3,0.5} &  antisolar point. \\%\midrule
apex\_points\_color 			 		& \ccbox{0.8,0.2,0.3} &  apex points. \\%\midrule
oc\_longitude\_color	     				& \ccbox{0.6,0.2,0.4} &  opposition/conjunction longitude. \\\midrule
%% Asterisms and ray helpers
asterism\_lines\_color      	 	 	& \ccbox{0.4,0.4,0.8} &  asterism lines. \\%\midrule
asterism\_names\_color      			& \ccbox{0.4,0.4,0.8} &  asterism names. \\%\midrule
rayhelper\_lines\_color      	 	 	& \ccbox{1.0,1.0,0.0} &  the ray helper lines. \\\midrule
%% Constellations
const\_lines\_color      				& \ccbox{0.2,0.2,0.6} &  constellation lines. \\%\midrule
const\_names\_color      				& \ccbox{0.4,0.6,0.9} &  constellation names. \\%\midrule
const\_boundary\_color   				& \ccbox{0.3,0.1,0.1} &  constellation boundaries. \\\midrule
%% DSO
dso\_label\_color                       & \ccbox{0.2,0.6,0.7} & deep-sky objects labels. \\%\midrule
dso\_circle\_color                      & \ccbox{1.0,0.7,0.2} & deep-sky objects symbols, if not of the types below. \\%\midrule
dso\_galaxy\_color                      & \ccbox{1.0,0.2,0.2} & galaxies symbols. \\%\midrule
dso\_radio\_galaxy\_color               & \ccbox{0.3,0.3,0.3} & radio galaxies symbols. \\%\midrule
dso\_active\_galaxy\_color              & \ccbox{1.0,0.5,0.2} & active galaxies symbols. \\%\midrule
dso\_interacting\_galaxy\_color         & \ccbox{0.2,0.8,1.0} & clusters of galaxies symbols. \\%\midrule
dso\_galaxy\_cluster\_color             & \ccbox{0.2,0.5,1.0} & interacting galaxies symbols. \\%\midrule
dso\_quasar\_color                      & \ccbox{1.0,0.2,0.2} & quasars symbols. \\%\midrule
dso\_possible\_quasar\_color            & \ccbox{1.0,0.2,0.2} & possible quasars symbols. \\%\midrule
dso\_bl\_lac\_color                     & \ccbox{1.0,0.2,0.2} & BL Lac objects symbols. \\%\midrule
dso\_blazar\_color                      & \ccbox{1.0,0.2,0.2} & blazars symbols. \\%\midrule
dso\_nebula\_color                      & \ccbox{0.1,1.0,0.1} & nebulae symbols. \\%\midrule
dso\_planetary\_nebula\_color           & \ccbox{0.1,1.0,0.1} & planetary nebulae symbols. \\%\midrule
dso\_reflection\_nebula\_color          & \ccbox{0.1,1.0,0.1} & reflection nebulae symbols. \\%\midrule
dso\_bipolar\_nebula\_color             & \ccbox{0.1,1.0,0.1} & bipolar nebulae symbols. \\%\midrule
dso\_emission\_nebula\_color            & \ccbox{0.1,1.0,0.1} & emission nebulae symbols. \\%\midrule
dso\_dark\_nebula\_color                & \ccbox{0.3,0.3,0.3} & dark nebulae symbols. \\%\midrule
dso\_hydrogen\_region\_color            & \ccbox{0.1,1.0,0.1} & hydrogen regions symbols. \\%\midrule
dso\_supernova\_remnant\_color          & \ccbox{0.1,1.0,0.1} & supernovae remnants symbols. \\%\midrule
dso\_supernova\_candidate\_color        & \ccbox{0.1,1.0,0.1} & supernova candidates symbols. \\%\midrule
dso\_supernova\_remnant\_cand\_color    & \ccbox{0.1,1.0,0.1} & supernova remnant candidates symbols. \\%\midrule
dso\_interstellar\_matter\_color        & \ccbox{0.1,1.0,0.1} & interstellar matter symbols. \\%\midrule
dso\_cluster\_with\_nebulosity\_color   & \ccbox{0.1,1.0,0.1} & clusters associated with nebulosity symbols. \\%\midrule
dso\_molecular\_cloud\_color            & \ccbox{0.1,1.0,0.1} & molecular clouds symbols. \\%\midrule
dso\_possible\_planetary\_nebula\_color & \ccbox{0.1,1.0,0.1} & possible planetary nebulae symbols. \\%\midrule
dso\_protoplanetary\_nebula\_color      & \ccbox{0.1,1.0,0.1} & protoplanetary nebulae symbols. \\%\midrule
dso\_cluster\_color                     & \ccbox{1.0,1.0,0.1} & star clusters symbols. \\%\midrule
dso\_open\_cluster\_color               & \ccbox{1.0,1.0,0.1} & open star clusters symbols. \\%\midrule
dso\_globular\_cluster\_color           & \ccbox{1.0,1.0,0.1} & globular star clusters symbols. \\%\midrule
dso\_stellar\_association\_color        & \ccbox{1.0,1.0,0.1} & stellar associations symbols. \\%\midrule
dso\_star\_cloud\_color                 & \ccbox{1.0,1.0,0.1} & star clouds symbols. \\%\midrule
dso\_star\_color                        & \ccbox{1.0,0.7,0.2} & star symbols. \\%\midrule
dso\_symbiotic\_star\_color             & \ccbox{1.0,0.7,0.2} & symbiotic star symbols. \\%\midrule
dso\_emission\_star\_color              & \ccbox{1.0,0.7,0.2} & emission-line star symbols. \\%\midrule
dso\_emission\_object\_color            & \ccbox{1.0,0.7,0.2} & emission objects symbols. \\%\midrule
dso\_young\_stellar\_object\_color      & \ccbox{1.0,0.7,0.2} & young stellar objects symbols. \\\midrule
%
star\_label\_color                      & \ccbox{0.4,0.3,0.5} & star labels. \\%\midrule
planet\_names\_color                    & \ccbox{0.5,0.5,0.7} & planet names. \\%\midrule
planet\_pointers\_color                 & \ccbox{1.0,0.3,0.3} & planet pointers. \\%\midrule
planet\_nomenclature\_color             & \ccbox{0.1,1.0,0.1} & planetary nomenclature hints and labels. \\\midrule
%
sso\_orbits\_color                      & \ccbox{0.7,0.2,0.2} & default orbits. \\%\midrule
major\_planet\_orbits\_color            & \ccbox{0.7,0.2,0.2} & major planet orbits. \\%\midrule
minor\_planet\_orbits\_color            & \ccbox{0.7,0.5,0.5} & minor planet orbits. \\%\midrule
dwarf\_planet\_orbits\_color            & \ccbox{0.7,0.5,0.5} & dwarf planet orbits. \\%\midrule
moon\_orbits\_color                     & \ccbox{0.7,0.2,0.2} & moon of planet orbits. \\%\midrule
cubewano\_orbits\_color                 & \ccbox{0.7,0.5,0.5} & cubewano orbits. \\%\midrule
plutino\_orbits\_color                  & \ccbox{0.7,0.5,0.5} & plutino orbits. \\%\midrule
sdo\_orbits\_color                      & \ccbox{0.7,0.5,0.5} & scattered disk object orbits. \\%\midrule
oco\_orbits\_color                      & \ccbox{0.7,0.5,0.5} & Oort cloud object orbits. \\%\midrule
sednoid\_orbits\_color                  & \ccbox{0.7,0.5,0.5} & Sednoid object orbits. \\%\midrule
interstellar\_orbits\_color             & \ccbox{1.0,0.6,1.0} & Interstellar object ``orbits'' (trajectories). \\%\midrule
comet\_orbits\_color                    & \ccbox{0.7,0.8,0.8} & comet orbits. \\%\midrule
mercury\_orbit\_color                   & \ccbox{0.5,0.5,0.5} & orbit of Mercury. \\%\midrule
venus\_orbit\_color                     & \ccbox{0.9,0.9,0.7} & orbit of Venus. \\%\midrule
earth\_orbit\_color                     & \ccbox{0.0,0.0,1.0} & orbit of Earth. \\%\midrule
mars\_orbit\_color                      & \ccbox{0.8,0.4,0.1} & orbit of Mars. \\%\midrule
jupiter\_orbit\_color                   & \ccbox{1.0,0.6,0.0} & orbit of Jupiter. \\%\midrule
saturn\_orbit\_color                    & \ccbox{1.0,0.8,0.0} & orbit of Saturn. \\%\midrule
uranus\_orbit\_color                    & \ccbox{0.0,0.7,1.0} & orbit of Uranus. \\%\midrule
neptune\_orbit\_color                   & \ccbox{0.0,0.3,1.0} & orbit of Neptune. \\\midrule
% other colors
object\_trails\_color                   & \ccbox{1.0,0.7,0.0} & planet trails. \\\midrule
custom\_marker\_color                   & \ccbox{0.1,1.0,0.1} & custom marker. \\\midrule
% Script colors
script\_console\_keyword\_color         & \ccbox{1.0,0.0,1.0} & syntax highlight for keywords in the script console. \\%\midrule
script\_console\_module\_color          & \ccbox{0.0,1.0,1.0} & syntax highlight for modules in the script console. \\%\midrule
script\_console\_comment\_color         & \ccbox{1.0,1.0,0.0} & syntax highlight for comments in the script console. \\%\midrule
script\_console\_function\_color        & \ccbox{0.0,1.0,0.0} & syntax highlight for functions in the script console. \\%\midrule
script\_console\_constant\_color        & \ccbox{1.0,0.5,0.5} & syntax highlight for constants in the script console. \\
\bottomrule
\end{longtable}

\subsection{\big[custom\_selected\_info\big]}
%\label{sec:config.ini:custom_selected_info}

You can fine-tune the bits of information to display for the selected object in this section. Set the entry to \emph{true} to display it.

\begin{longtable}{l|l|l}\toprule
\emph{ID} & \emph{Type} & \emph{Description}\\\midrule
flag\_show\_absolutemagnitude & bool & absolute magnitude for objects.\\%\midrule
flag\_show\_altaz             & bool & horizontal coordinates for objects.\\%\midrule
flag\_show\_catalognumber     & bool & catalog designations for objects.\\%\midrule
flag\_show\_distance          & bool & distance to object.\\%\midrule
flag\_show\_extra             & bool & extra info for object.\\%\midrule
flag\_show\_hourangle         & bool & hour angle for object.\\%\midrule
flag\_show\_magnitude         & bool & magnitude for object.\\%\midrule
flag\_show\_name              & bool & common name for object.\\%\midrule
flag\_show\_radecj2000        & bool & equatorial coordinates (J2000.0 frame) of object.\\%\midrule
flag\_show\_radecofdate       & bool & equatorial coordinates (of date) of object.\\%\midrule
flag\_show\_galcoord          & bool & galactic coordinates (System~II) of object.\\%\midrule
flag\_show\_supergalcoord     & bool & supergalactic coordinates of object.\\%\midrule
flag\_show\_eclcoordj2000     & bool & ecliptic coordinates (J2000.0 frame) of object.\\%\midrule
flag\_show\_eclcoordofdate    & bool & ecliptic coordinates (of date) of object.\\%\midrule
flag\_show\_type              & bool & type of object.\\%\midrule
flag\_show\_size              & bool & size of object.\\%\midrule
flag\_show\_sidereal\_time    & bool & sidereal time.\\%\midrule
flag\_show\_rts\_time         & bool & rise, transit and set times.\\%\midrule
flag\_show\_velocity          & bool & velocity of object.\\%\midrule
flag\_show\_constellation     & bool & show 3-letter IAU\index{IAU} constellation label.\\\bottomrule
\end{longtable}

\subsection{\big[custom\_time\_correction\big]}
%\label{sec:config.ini:custom_time_correction}

Stellarium allows experiments with $\Delta T$. See \ref{sec:Concepts:DeltaT} for details.

\noindent%
\begin{tabularx}{\textwidth}{l|l|X}\toprule
\emph{ID}    & \emph{Type} & \emph{Description}\\\midrule
coefficients & [float,float,float] & Coefficients $a, b, c$ for custom equation of $\Delta T$\\%\midrule
ndot & float & n-dot value for custom equation of $\Delta T$\\%\midrule
year & int   & Year for custom equation of $\Delta T$\\\bottomrule
\end{tabularx}

\subsection{\big[devel\big]}
%\label{sec:config.ini:devel}

This section is for developers only. 
%% TODO: Still it has to be documented!

\noindent%
\begin{tabularx}{\textwidth}{l|l|X}\toprule
\emph{ID}              & \emph{Type} & \emph{Description}\\\midrule
convert\_dso\_catalog        & bool & Set to \emph{true} to convert file \file{catalog.txt} 
                                      into file \file{catalog.dat}. Default value: \emph{false}.\\%\midrule
convert\_dso\_decimal\_coord & bool & Set to \emph{true} to use decimal values for coordinates 
                                      in source catalog. Default value: \emph{true}.\\\bottomrule
\end{tabularx}

\subsection{\big[dso\_catalog\_filters\big]}
%\label{sec:config.ini:dso_catalog_filters}
In this section you can fine-tune which of the deep-sky catalogs should be selected on startup.

\begin{longtable}{l|l|l|p{90mm}}\toprule
\emph{ID} & \emph{Type} & \emph{Default} & \emph{Description}\\\midrule
flag\_show\_ngc  & bool & true  & New General Catalogue (NGC). \\%\midrule
flag\_show\_ic   & bool & true  & Index Catalogue (IC). \\%\midrule
flag\_show\_m    & bool & true  & Messier Catalog (M). \\%\midrule
flag\_show\_c    & bool & false & Caldwell Catalogue (C). \\%\midrule
flag\_show\_b    & bool & false & Barnard Catalogue (B). \\%\midrule
flag\_show\_sh2  & bool & false & Sharpless Catalogue (Sh-II). \\%\midrule
flag\_show\_vdb  & bool & false & Van den Bergh Catalogue of reflection nebulae (VdB). \\%\midrule
flag\_show\_rcw  & bool & false & The RCW catalogue of H$\alpha$-emission regions in the southern Milky Way. \\%\midrule
flag\_show\_lbn  & bool & false & Lynds' Catalogue of Bright Nebulae (LBN). \\%\midrule
flag\_show\_ldn  & bool & false & Lynds' Catalogue of Dark Nebulae (LDN). \\%\midrule
flag\_show\_cr   & bool & false & Collinder Catalogue (Cr). \\%\midrule
flag\_show\_mel  & bool & false & Melotte Catalogue of Deep Sky Objects (Mel).  \\%\midrule
flag\_show\_pgc  & bool & false & HYPERLEDA. I. Catalog of galaxies (PGC). \\%\midrule
flag\_show\_ced  & bool & false & Cederblad Catalog of bright diffuse Galactic nebulae (Ced). \\%\midrule
flag\_show\_ugc  & bool & false & The Uppsala General Catalogue of Galaxies (UGC). \\%\midrule
flag\_show\_arp  & bool & false & Atlas of Peculiar Galaxies (Arp). \\%\midrule
flag\_show\_vv   & bool & false & The Catalogue of Interacting Galaxies (VV). \\%\midrule
flag\_show\_pk   & bool & false & The Catalogue of Galactic Planetary Nebulae (PK). \\%\midrule
flag\_show\_png  & bool & false & The Strasbourg-ESO Catalogue of Galactic Planetary Nebulae (PN G). \\%\midrule
flag\_show\_snrg & bool & false & A catalogue of Galactic supernova remnants (SNR G). \\%\midrule
flag\_show\_aco  & bool & false & A Catalog of Rich Clusters of Galaxies (ACO). \\%\midrule
flag\_show\_hcg  & bool & false & A Hickson Compact Group (HCG). \\%\midrule
flag\_show\_abell  & bool & false & Abell Catalog of Planetary Nebulae (Abell). \\%\midrule
flag\_show\_eso  & bool & false & ESO/Uppsala Survey of the ESO(B) Atlas (ESO). \\\bottomrule
\end{longtable}

\subsection{\big[dso\_type\_filters\big]}
%\label{sec:config.ini:dso_type_filters}
In this section you can fine-tune which types of the deep-sky objects should be selected on startup.

\noindent%
\begin{tabularx}{\textwidth}{l|l|l|X}\toprule
\emph{ID} & \emph{Type} & \emph{Default} & \emph{Description}\\\midrule
flag\_show\_galaxies              & bool & true & display galaxies.\\%\midrule
flag\_show\_active\_galaxies      & bool & true & display active galaxies. \\%\midrule
flag\_show\_interacting\_galaxies & bool & true & display interacting galaxies.  \\%\midrule
flag\_show\_clusters              & bool & true & display star clusters.  \\%\midrule
flag\_show\_bright\_nebulae       & bool & true & display bright nebulae.  \\%\midrule
flag\_show\_dark\_nebulae         & bool & true & display dark nebulae.  \\%\midrule
flag\_show\_planetary\_nebulae    & bool & true & display planetary nebulae.  \\%\midrule
flag\_show\_hydrogen\_regions     & bool & true & display hydrogen regions. \\%\midrule
flag\_show\_supernova\_remnants   & bool & true & display supernovae remnants. \\%\midrule
flag\_show\_galaxy\_clusters      & bool & true & display clusters of galaxies. \\%\midrule
flag\_show\_other                 & bool & true & display other deep-sky objects.  \\\bottomrule
\end{tabularx}

\subsection{\big[gui\big]}
%\label{sec:config.ini:gui}

This section includes settings for the graphical user interface. See sections \ref{sec:gui:configuration:extras}--\ref{sec:gui:configuration:tools} for most of these keys.

\begin{longtable}{p{50mm}|l|l|p{55mm}}
\toprule
\emph{ID} & \emph{Type} & \emph{Default} & \emph{Description}\\\midrule
flag\_show\_gui                 & bool   & true & If \emph{true}, display GUI.\\\midrule
screen\_font\_size & int  & 13 & Sets the font size for screen texts. Typical value: \emph{13..15}\\%\midrule
gui\_font\_size  & int    & 13 & Sets the dialog font size. Typical value: \emph{10..20}\\%\midrule
base\_font\_name & string & Verdana     & Selects the name \\
                 &        & (Windows)            & for base font\\
                 &        & DejaVu Sans & \\
                 &        & (others)            & \\%\midrule
%safe\_font\_name               & string &      & Selects the name for safe font, e.g. \emph{Verdana}\\\midrule
base\_font\_file                & string &      & Selects the name for font file, e.g. \emph{DejaVuSans.ttf}\\
flag\_font\_selection           & bool  & false & Show GUI elements for font selection.\\\midrule
%% TODO: Is this required? timeout=0 also keeps cursor visible.
flag\_mouse\_cursor\_timeout    & bool  & true  & Set to \emph{false} if you want to have cursor visible at all times.\\%\midrule
mouse\_cursor\_timeout          & float & 10    & Set to \emph{0} if you want to keep the mouse cursor visible at all times. 
                                                  non-0 values mean the cursor will be hidden after that many seconds of inactivity\\\midrule
%
flag\_show\_boundaries\_button         & bool  & false & Show a button to toggle constellation boundaries\\
flag\_show\_asterism\_lines\_button    & bool  & false & Show a button to toggle asterism lines\\
flag\_show\_asterism\_labels\_button   & bool  & false & Show a button to toggle asterism labels\\
flag\_show\_ecliptic\_grid\_button     & bool  & false & Show a button to toggle ecliptic grid\\
flag\_show\_icrs\_grid\_button         & bool  & false & Show a button to toggle ICRS (equatorial J2000.0) grid\\
flag\_show\_galactic\_grid\_button     & bool  & false & Show a button to toggle galactic grid\\
flag\_show\_goto\_selected\_button     & bool  & true  & Show a button to center to selected object\\
flag\_show\_nightmode\_button          & bool  & true  & Show a button to toggle night mode\\
flag\_show\_nebulae\_ \mbox{background\_button} & bool & false & Show a button to toggle images for deep-sky objects\\%\midrule
flag\_show\_flip\_buttons              & bool  & false & Show the image flipping buttons in the main toolbar \\%\midrule
flag\_show\_dss\_button                & bool  & false & Show a button to toggle Digitized Sky Survey (DSS)\\%\midrule
flag\_show\_hips\_button               & bool  & false & Show a button to toggle HiPS survey display\\%\midrule
flag\_show\_bookmark\_button           & bool  & false & Show a button to show and edit bookmarks\\
flag\_show\_fullscreen\_button         & bool  & true  & Show a button to toggle fullscreen view\\
flag\_show\_quit\_button               & bool  & true  & Show a button to quit via button press\\
flag\_show\_buttons\_background        & bool  & true  & If \emph{true}, use background under buttons on bottom bar.\\\midrule
%
selected\_object\_info            & string & all  & Values: \emph{all}, \emph{short}, \emph{none}, 
                                                    and \emph{custom} (since V0.12.0; see~\ref{sec:gui:configuration:info}).\\\midrule
custom\_marker\_size             & float   & 5.0 & Size of custom marker.\\%\midrule
custom\_marker\_radius\_limit    & int   & 15 & Limit the click radius for mouse cursor position in any direction for removing of closest custom marker.\\\midrule
%% NO LONGER HERE!
%day\_key\_mode                  & string & Specifies the amount of time which is added and subtracted when the 
%                                                    {[} {]} - and = keys are pressed - calendar days, or sidereal days. 
%                                                    This option only makes sense for Digitalis planetariums. %% WHY?
%                                                    Values: \emph{calendar} or \emph{sidereal}\\\midrule
auto\_hide\_horizontal\_toolbar & bool   & true  & Set to \emph{true} if you want auto hide the horizontal toolbar.\\%\midrule
auto\_hide\_vertical\_toolbar   & bool   & true  & Set to \emph{true} if you want auto hide the vertical toolbar.\\%\midrule
flag\_use\_window\_transparency & bool   & false & If \emph{false}, show menu bars opaque\\%\midrule
flag\_show\_datetime            & bool   & true  & display date and time in the bottom bar\\%\midrule
flag\_time\_jd                  & bool   & false & use JD format for time in the bottom bar\\%\midrule
flag\_show\_tz                  & bool   & false & show time zone info in the bottom bar\\%\midrule
flag\_show\_location            & bool   & true  & display location in the bottom bar\\%\midrule
flag\_show\_fps                 & bool   & true  & show at how many frames per second Stellarium is rendering\\%\midrule
flag\_show\_fov                 & bool   & true  & show how many degrees your vertical field of view is\\%\midrule
flag\_fov\_dms                  & bool   & false & use DMS format for FOV in the bottom bar\\\midrule
%
flag\_show\_decimal\_degrees    & bool   & false & If \emph{true}, use decimal degrees for coordinates\\\midrule
flag\_use\_azimuth\_from\_south & bool   & false & If \emph{true}, calculate azimuth from south towards west 
                                                   (as in some astronomical literature)\\\midrule
flag\_surface\_brightness\_arcsec  & bool  & false & Toggle usage the measure unit $^{mag}/_\square{'}$ ($^{mag}/arcmin^2$) or 
                                                     $^{mag}/_\square{''}$ ($^{mag}/arcsec^2$) for the surface brightness of deep-sky objects.\\%\midrule
flag\_surface\_brightness\_short  & bool   & false & Toggle usage the short notation for the surface brightness of deep-sky objects.\\\midrule
flag\_enable\_kinetic\_scrolling & bool  & true  & If \emph{true}, use kinetic scrolling in the GUI\footnote{%
                                                   Scroll by dragging in the window directly, not a slider handle. Scrolling stops after a short time.}.\\\midrule
pointer\_animation\_speed 		& float  & 1.0 	 & Animation speed of pointers.\\\midrule
%
gpsd\_hostname                  & string & localhost & hostname of server running \program{gpsd} (non-Windows only)\\%\midrule
gpsd\_port                      & int    & 2947   & port number of \program{gpsd} (non-Windows only)\\%\midrule
gps\_interface                  & string &  COM3  & Port number of serial/USB GPS device. (May look like \texttt{ttyUSB0} on Linux,
                                                    if you really don't run \program{gpsd}.)\\%\midrule
gps\_baudrate                   & int    &  4800  & baudrate of serial/USB GPS device (for direct serial connection only)\\\bottomrule
\end{longtable}

\subsection{\big[init\_location\big]}
%\label{sec:config.ini:init_location}

\begin{tabularx}{\textwidth}{l|l|X}\toprule
\emph{ID} & \emph{Type} & \emph{Description}\\\midrule
landscape\_name   & string & Sets the landscape you see. Built-in options are \emph{garching, geneva, grossmugl, guereins, 
                             hurricane, jupiter, mars, moon, neptune, ocean, saturn, trees, uranus, zero}.\\%\midrule
location          & string & Name of location on which to start stellarium.\\%\midrule
last\_location    & string & Coordinates of last used location in stellarium.\\\bottomrule
\end{tabularx}

\subsection{\big[landscape\big]}
%% Totally crazy, one label, and tex limits are exceeded for tex4ht. 
\ifhtlatex
\else
\label{sec:configini:landscape}
\fi

\begin{longtable}{l|l|p{80mm}}\toprule
\emph{ID}                            & \emph{Type} & \emph{Description}\\\midrule
atmosphere\_fade\_duration                 & float & Sets the time (seconds) it takes for the atmosphere to fade when de-selected\\%\midrule
flag\_landscape                            & bool  & Set to \emph{false} if you don't want to see the landscape at all\\%\midrule
flag\_fog                                  & bool  & Set to \emph{false} if you don't want to see fog on start-up\\%\midrule
flag\_atmosphere                           & bool  & Set to \emph{false} if you don't want to see atmosphere on start-up\\%\midrule
flag\_landscape\_sets\_location            & bool  & Set to \emph{true} if you want Stellarium to modify the observer location 
                                                     when a new landscape is selected (changes planet and longitude/latitude/altitude 
                                                     if location data is available in the landscape.ini file)\\%\midrule
minimal\_brightness                        & float & Set minimal brightness for landscapes. [0\ldots1] Typical value: \emph{0.01}\\%\midrule
atmosphereybin						       & int   & Set atmosphere binning coefficient for axis Y.\\%\midrule
flag\_minimal\_brightness                  & bool  & Set to \emph{true} to use minimal brightness for landscape.\\%\midrule
flag\_landscape\_sets\_minimal\_brightness & bool  & Set to \emph{true} to use value for minimal brightness for landscape from landscape settings.\\%\midrule
flag\_enable\_illumination\_layer          & bool  & Set to \emph{true} to use illumination layer for landscape.\\%\midrule
flag\_enable\_labels                       & bool  & Set to \emph{true} to use landscape labels from gazetteer layer.\\%\midrule
atmospheric\_extinction\_coefficient       & float & Set atmospheric extinction coefficient $k$ [mag/airmass]\\%\midrule
temperature\_C                             & float & Set atmospheric temperature [Celsius]\\%\midrule
pressure\_mbar                             & float & Set atmospheric pressure [mbar]\\%\midrule
cache\_size\_mb                            & int   & Set the cache size for landscapes [megabytes]. Default: 100\\\bottomrule
\end{longtable}

\subsection{\big[localization\big]}
%\label{sec:config.ini:localization}

\begin{tabularx}{\textwidth}{l|l|X}\toprule
\emph{ID}         & \emph{Type} & \emph{Description}\\\midrule
sky\_culture          & string & Sets the sky culture to use. E.g. \emph{western, polynesian, egyptian, chinese, 
                                 lakota, navajo, inuit, korean, norse, tupi, maori, aztec, sami}.\\%\midrule
sky\_locale           & string & Sets language used for names of objects in the sky (e.g. planets). 
                                 The value is a short locale code, e.g. \emph{en, de, en\_GB}\\%\midrule
app\_locale           & string & Sets language used for Stellarium's user interface. 
                                  The value is a short locale code, e.g. \emph{en, de, en\_GB}.\\%\midrule
time\_zone            & string & Sets the time zone. Valid values: \emph{system\_default}, 
                                 or some region/location combination, e.g. \emph{Pacific/Marquesas}\\%\midrule
time\_display\_format & string & time display format: can be \emph{system\_default}, \emph{24h} or \emph{12h}.\\%\midrule
date\_display\_format & string & date display format: can be \emph{system\_default}, \emph{mmddyyyy}, \emph{ddmmyyyy}, \emph{yyyymmdd} (ISO8601), \emph{wwmmddyyyy}, \emph{wwddmmyyyy}, \emph{wwyyyymmdd}.\\\bottomrule
\end{tabularx}

\subsection{\big[main\big]}
%\label{sec:config.ini:main}

\begin{tabularx}{\textwidth}{l|l|X}\toprule
\emph{ID}                 & \emph{Type} & \emph{Description}\\\midrule
invert\_screenshots\_colors & bool   & If \emph{true}, Stellarium will saving the screenshorts with inverted colors.\\%\midrule
screenshot\_dir             & string & Path for saving screenshots\\%\midrule
screenshot\_format          & string & File format. One of \texttt{png|jpg|jpeg|tif|tiff| webp|pbm|pgm|ppm|xbm|xpm|ico}\\
screenshot\_custom\_size    & bool   & \emph{true} if you want to specify particular dimensions next:\\
screenshot\_custom\_width   & int    & custom width of screenshots. Max.\ available size is hardware dependent!\\
screenshot\_custom\_height  & int    & custom height of screenshots. Max.\ available size is hardware dependent!\\\midrule
version                     & string & Version of Stellarium. This parameter may be used to detect necessary changes in config.ini file, do not edit.\\%\midrule
use\_separate\_output\_file & bool   & Set to \emph{true} if you want to create a new file for script output for each start of Stellarium\\%\midrule
restore\_defaults           & bool   & If \emph{true}, Stellarium will restore default settings at startup. 
                                       This is equivalent to calling Stellarium with the \command{--restore-defaults} option.\\%\midrule
ignore\_opengl\_warning     & bool   & Set to \emph{true} if you don't want to see OpenGL warnings for each start of Stellarium.\\%\midrule
check\_requirements         & bool   & Set to \emph{false} if you want to disable and permanently ignore checking hardware requirements at startup. 
                                       Expect problems if hardware is below requirements!\\
geoip\_api\_url             & string & URL for JSON interface of GeoIP service.\\
\bottomrule
\end{tabularx}

\subsection{\big[navigation\big]}
%\label{sec:config.ini:navigation}

This section controls much of the look\&feel of Stellarium. Be careful if you change something here.

%% TODO: improve description of the various zoom settings.

\begin{longtable}{l|l|p{77mm}}\toprule
\emph{ID}              & \emph{Type} & \emph{Description}\\\midrule
preset\_sky\_time               & float  & Preset sky time used by the dome version. Unit is Julian Day. Typical value: \emph{2451514.250011573}\\%\midrule
startup\_time\_mode             & string & Set the start-up time mode, can be \emph{actual} (start with current real world time), 
                                           or \emph{Preset} (start at time defined by preset\_sky\_time)\\%\midrule
flag\_enable\_zoom\_keys        & bool   & Set to \emph{false} if you want to disable the zoom\\%\midrule
flag\_manual\_zoom              & bool   & Set to \emph{false} for normal zoom behaviour as described in this guide. 
                                           When set to true, the auto zoom feature only moves in a small amount and must be pressed many times\\%\midrule
flag\_enable\_move\_keys        & bool   & Set to \emph{false} if you want to disable the arrow keys\\%\midrule
flag\_enable\_mouse\_navigation & bool   & Set to \emph{false} if you want to disable the mouse navigation.\\%\midrule
init\_fov                       & float  & Initial field of view, in degrees, typical value: ''60''.\\%\midrule
min\_fov                        & float  & Minimal field of view, in degrees, typical value: ''0.001389'' (5").\\%\midrule
init\_view\_pos                 & floats & Initial viewing direction. This is a vector with x,y,z-coordinates. x being N-S (S +ve), 
                                           y being E-W (E +ve), z being up-down (up +ve). Thus to look South at the horizon use \emph{1,0,0}. 
                                           To look Northwest and up at $45\degree$, use \emph{-1,-1,1} and so on.\\%\midrule
auto\_move\_duration            & float  & Duration (seconds) for the program to move to point at an object when the space bar is pressed. 
                                           Typical value: \emph{1.4}\\%\midrule
mouse\_zoom                     & float  & Sets the mouse zoom amount (mouse-wheel)\\%\midrule
move\_speed                     & float  & Sets the speed of movement\\%\midrule
zoom\_speed                     & float  & Sets the zoom speed\\%\midrule
viewing\_mode                   & string & If set to \emph{horizon}, the viewing mode simulate an alt/azi mount, 
                                           if set to \emph{equator}, the viewing mode simulates an equatorial mount\\%\midrule
flag\_manual\_zoom              & bool   & Set to \emph{true} if you want to auto-zoom in incrementally.\\%\midrule
auto\_zoom\_out\_resets\_direction & bool & Set to \emph{true} if you want to auto-zoom restoring direction.\\%\midrule
time\_correction\_algorithm     & string & Algorithm of DeltaT correction.\\\bottomrule %% TODO: list values! And write ref. chapter!
\end{longtable}

\subsection{\big[plugins\_load\_at\_startup\big]}
%\label{sec:config.ini:plugins_load_at_startup}

This section lists which plugins are loaded at startup (those with
\emph{true} values). Each plugin can add another section into this
file with its own content, which is described in the respective plugin
documentation, see \ref{ch:Plugins}. You activate loading of plugins
in the \key{F2} settings dialog, tab ``Plugins''. After selection of
which plugins to load, you must restart Stellarium.

\begin{longtable}{l|l|p{80mm}}
\toprule
\emph{ID} & \emph{Type} & \emph{Description}\\\midrule
AngleMeasure          & bool & \ref{sec:plugins:AngleMeasure} Angle Measure plugin\\%\midrule
ArchaeoLines          & bool & \ref{sec:plugin:ArchaeoLines} ArchaeoLines plugin\\%\midrule
CompassMarks          & bool & \ref{sec:plugins:CompassMarks} Compass Marks plugin\\%\midrule
MeteorShowers         & bool & \ref{sec:plugins:MeteorShowers} Meteor Showers plugin \\%\midrule
Exoplanets            & bool & \ref{sec:plugins:Exoplanets} Exoplanets plugin \\%\midrule
Observability         & bool & \ref{sec:plugins:Observability} Observability Analysis\\%\midrule
Oculars               & bool & \ref{sec:plugins:Oculars} Oculars plugin \\%\midrule
Pulsars               & bool & \ref{sec:plugins:Pulsars} Pulsars plugin \\%\midrule
Quasars               & bool & \ref{sec:plugins:Quasars} Quasars plugin \\%\midrule
RemoteControl         & bool & \ref{sec:plugin:RemoteControl} Remote Control plugin \\%\midrule
RemoteSync            & bool & \ref{sec:plugin:RemoteSync} Remote Sync plugin \\%\midrule
Satellites            & bool & \ref{sec:plugins:Satellites} Satellites plugin \\%\midrule
SolarSystemEditor     & bool & \ref{sec:plugins:SolarSystemEditor} Solar System Editor plugin\\%\midrule
Supernovae            & bool & \ref{sec:plugins:HistoricalSupernovae} Historical Supernovae plugin \\%\midrule
TelescopeControl      & bool & \ref{sec:plugins:TelescopeControl} Telescope Control plugin \\%\midrule
TextUserInterface     & bool & \ref{sec:plugins:TextUserInterface} Text User Interface plugin \\%\midrule
%TimeZoneConfiguration & bool & \ref{sec:plugins:TimeZoneControl} Time Zone plugin \\\midrule
Novae                 & bool & \ref{sec:plugins:BrightNovae} Bright Novae plugin \\%\midrule
Scenery3dMgr          & bool & \ref{ch:scenery3d} Scenery 3D plugin \\\bottomrule
\end{longtable}

\subsection{\big[projection\big]}
%\label{sec:config.ini:projection}

This section contains the projection of your choice and several
advanced settings important if you run Stellarium on a single screen,
multi-projection, dome projection, or other setups.

\begin{longtable}{l|l|p{80mm}}\toprule
\emph{ID} & \emph{Type} & \emph{Description}\\\midrule
type                       & string & Sets projection mode. Values: \emph{ProjectionPerspective,
                                      ProjectionEqualArea, ProjectionStereographic, ProjectionFisheye,
                                      ProjectionHammer, ProjectionCylinder, ProjectionMercator,
                                      ProjectionOrthographic, ProjectionMiller}, or \emph{ProjectionSinusoidal}.\\%\midrule
flip\_horz                   & bool & \\%\midrule
flip\_vert                   & bool & \\%\midrule
viewport                   & string & How the view-port looks. Values: \emph{none} (regular rectangular screen), 
                                      \emph{disk} (circular screen, useful for planetarium setup).\\%\midrule
viewportMask               & string & How the view-port looks. Values: \emph{none}.\\%\midrule %% TODO: DESCRIBE HOW TO USE!
viewport\_fov\_diameter     & float & \\%\midrule
viewport\_x                 & float & Usually 0. \\%\midrule
viewport\_y                 & float & Usually 0. \\%\midrule
viewport\_width             & float & \\%\midrule
viewport\_height            & float & \\%\midrule
viewport\_center\_x         & float & Usually half of viewport\_width. \\%\midrule
viewport\_center\_y         & float & Usually half of viewport\_height. \\%\midrule
viewport\_center\_offset\_x & float & [-0.5\ldots+0.5] Usually 0. \\%\midrule
viewport\_center\_offset\_y & float & [-0.5\ldots+0.5] Use negative values to move the horizon lower. \\
\bottomrule
\end{longtable}

\subsection{\big[proxy\big]}
%\label{sec:config.ini:proxy}
This section has setting for connection to network through proxy server (proxy will be using when host of proxy is filled).

\noindent%
\begin{tabularx}{\textwidth}{l|l|X}\toprule
\emph{ID}  & \emph{Type} & \emph{Description}\\\midrule
host\_name & string & Name of host for proxy. E.g. \emph{proxy.org}\\%\midrule
type	   & string & Type of proxy. E.g. \emph{socks}\\%\midrule
port       & int    & Port of proxy. E.g. \emph{8080}\\%\midrule
user       & string & Username for proxy. E.g. \emph{michael\_knight}\\%\midrule
password   & string & Password for proxy. E.g. \emph{xxxxx}\\\bottomrule
\end{tabularx}

\subsection{\big[scripts\big]}
%\label{sec:config.ini:scripts}

\begin{tabularx}{\textwidth}{l|l|l|X}\toprule
\emph{ID}                  & \emph{Type} & \emph{Default} & \emph{Description}\\\midrule
startup\_script                            & string & startup.ssc & name of script executed on program start\\
flag\_script\_allow\_write\_absolute\_path & bool   & false       & set true to let scripts store files to absolute pathnames.
                                                                    This may pose a security risk if you run scripts from other authors
                                                                    without checking what they are doing.\\\bottomrule 
%% These are unused as on 0.15pre.
%flag\_script\_allow\_ui        & bool &\\\midrule
%scripting\_allow\_write\_files & bool &\\\bottomrule
\end{tabularx}

\subsection{\big[search\big]}
%\label{sec:config.ini:search}

\begin{tabularx}{\textwidth}{l|l|X}\toprule
\emph{ID} & \emph{Type} & \emph{Description}\\\midrule
flag\_search\_online & bool   & If \emph{true}, Stellarium will be use SIMBAD for search.\\%\midrule
simbad\_server\_url  & string & URL for SIMBAD mirror\\%\midrule
flag\_start\_words   & bool   & If \emph{false}, Stellarium will be search phrase only from start of words\\%\midrule
coordinate\_system   & string & Specifies the coordinate system. 
                                \emph{Possible values:} equatorialJ2000, equatorial, horizontal, galactic. \emph{Default value:} equatorialJ2000.\\
\bottomrule
\end{tabularx}

\subsection{\big[spheric\_mirror\big]}
%\label{sec:config.ini:spheric_mirror}

Stellarium can be used in planetarium domes. You can use a projector with a hemispheric mirror with geometric properties given in this section. 
Note: These functions are only rarely used or tested, some may not work as expected.
%% TODO: Make sure we can remove this note! Maybe add a chapter in the ``Advanced setup'' part?
 
\begin{longtable}{l|l|l|p{68mm}}\toprule
\emph{ID} & \emph{Type} & \emph{Default}&\emph{Description}\\\midrule
flip\_horz             & bool  &true & Flip the projection horizontally\\%\midrule
flip\_vert             & bool  &false& Flip the projection vertically\\%\midrule
projector\_alpha       & float &0& This parameter controls the properties of the spheric mirror projection mode.\\%\midrule
projector\_gamma       & float && This parameter controls the properties of the spheric mirror projection mode.\\%\midrule
projector\_delta       & float &-1e100& This parameter controls the properties of the spheric mirror projection mode.\\%\midrule
projector\_phi         & float &0& This parameter controls the properties of the spheric mirror projection mode.\\%\midrule
projector\_position\_x & float &0& \\%\midrule
projector\_position\_y & float &1& \\%\midrule
projector\_position\_z & float &-0.2& \\%\midrule
mirror\_position\_x    & float &0& \\%\midrule
mirror\_position\_y    & float &2& \\%\midrule
mirror\_position\_z    & float &0& \\%\midrule
image\_distance\_div\_height&float&-1e100\\%\midrule
mirror\_radius         & float &0.25& \\%\midrule
dome\_radius           & float &2.5& \\%\midrule
custom\_distortion\_file& string& \\%\midrule
texture\_triangle\_base\_length&float& \\%\midrule
zenith\_y              & float &0.125& deprecated\\%\midrule
scaling\_factor        & float &0.8  & deprecated \\%\midrule
%% GZ I found some very strange names in StelViewportEffect.cpp
distorter\_max\_fov    & float & 175.0&Set the maximum field of view for the spheric mirror distorter in degrees. Typical value: \emph{180}\\%\midrule
%% TODO: MAKE SURE THIS IS STILL USED?
%flag\_use\_ext\_framebuffer\_object & bool & Some video hardware incorrectly claims to support some GL extension, GL\_FRAMEBUFFER\_EXTEXT. If, when using the spheric mirror distorter the
%frame rate drops to a very low value (e.g. 0.1 FPS), set this parameter to \emph{false} to tell Stellarium to ignore the claim of the video driver that it can use this extension\\\midrule
viewportCenterWidth    & float && projected image width, pixels\\%\midrule %% TODO Value has buggy name. Test functionality/Discuss!
viewportCenterHeight   & float && projected image height, pixels\\%\midrule %% TODO Value has buggy name. Test functionality/Discuss!
viewportCenterX        & float && projected image center X, pixels\\%\midrule
viewportCenterY        & float && projected image center Y, pixels\\
\bottomrule 
\end{longtable}

\subsection{\big[stars\big]}
%\label{sec:config.ini:stars}

This section controls how stars are rendered.
%% TODO: Explain what we mean with "magnitude conversion routine"!

\noindent%
\begin{tabularx}{\textwidth}{l|l|X}\toprule
\emph{ID}                & \emph{Type} & \emph{Description}\\\midrule
relative\_scale          & float       & relative size of bright and faint stars. Higher values mean that bright stars 
                                         are comparitively larger when rendered. Typical value: \emph{1.0}\\%\midrule
absolute\_scale          & float       & Changes how large stars are rendered. larger value lead to larger depiction. Typical value: \emph{1.0}\\%\midrule
star\_twinkle\_amount    & float       & amount of twinkling. Typical value: \emph{0.3}\\%\midrule
flag\_star\_twinkle      & bool        & \emph{true} to allow twinkling (but only when atmosphere is active!).\\%\midrule
flag\_forced\_twinkle    & bool        & \emph{true} to allow twinkling stars even without atmosphere. This is obvious nonsense and contrary to nature, but some users seem to like it.\\%\midrule
flag\_star\_halo         & bool        & \emph{false} to not draw a halo around the brightest stars. This looks poor, but some users seem to prefer it.\\%\midrule
mag\_converter\_max\_fov & float       & maximum field of view for which the magnitude conversion routine is used. Typical value: \emph{90.0}.\\%\midrule
mag\_converter\_min\_fov & float       & minimum field of view for which the magnitude conversion routine is used. Typical value: \emph{0.001}.\\%\midrule
labels\_amount           & float       & amount of labels. Typical value: \emph{3.0}\\%\midrule
init\_bortle\_scale      & int         & initial value of light pollution on the Bortle scale. Typical value: \emph{3}.\\\bottomrule
\end{tabularx}

\subsection{\big[tui\big]}
%\label{sec:config.ini:tui}

The built-in text user interface (TUI) plugin (see chapter~\ref{sec:plugins:TextUserInterface}) is most useful for planetariums. You can even configure a system shutdown command. 
For historical reasons, the section is not called \texttt{[TextUserInterface]} but simply \texttt{[tui]}.
%% TODO: Move to TUI plugin chapter?

\noindent%
\begin{tabularx}{\textwidth}{l|l|l|X}\toprule
\emph{ID} & \emph{Type} &\emph{Default}& \emph{Description}\\\midrule
%flag\_enable\_tui\_menu & bool & Enables or disables the TUI menu\\\midrule
tui\_font\_size                   &float    & 15       & Font size for TUI.\\%\midrule
tui\_font\_color                  &floatRGB &0.3,1,0.3 & Font color for TUI.\\%\midrule
flag\_show\_gravity\_ui           & bool    & false    & Set to \emph{true} to enable gravity mode for UI (for dome projection)\\%\midrule
flag\_show\_tui\_datetime         & bool    & false    & Set to \emph{true} if you want to see a date and time label suited for dome projections\\%\midrule
flag\_show\_tui\_short\_obj\_info & bool    & false    & Set to \emph{true} if you want to see object info suited for dome projections\\%\midrule
% TODO: DESCRIBE!
tui\_admin\_shutdown\_command       & string  && e.g.\ for Linux: \command{shutdown --poweroff +2} \\\bottomrule
\end{tabularx}

\subsection{\big[video\big]}
%\label{sec:config.ini:video}

\begin{tabularx}{\textwidth}{l|l|X}\toprule
\emph{ID}  & \emph{Type} & \emph{Description}\\\midrule
fullscreen       & bool   & If \emph{true}, Stellarium will start up in full-screen mode, else windowed mode\\%\midrule
screen\_w        & int    & Display width when in windowed mode. Value in pixels, e.g. \emph{1024}\\%\midrule
screen\_h        & int    & Display height when in windowed mode. Value in pixels, e.g. \emph{768}\\%\midrule
screen\_x        & int    & Horizontal position of the top-left corner in windowed mode. Value in pixels, e.g. \emph{0}\\%\midrule
screen\_y        & int    & Vertical   position of the top-left corner in windowed mode. Value in pixels, e.g. \emph{0}\\%\midrule
viewport\_effect & string & This is used when the spheric mirror display mode is activated. Values include \emph{none} and \emph{sphericMirrorDistorter}.\\%\midrule
minimum\_fps     & int    & Sets the minimum number of frames per second to display at (hardware performance permitting)\\%\midrule
maximum\_fps     & int    & Sets the maximum number of frames per second to display at. This is useful to reduce power consumption in laptops.\\\bottomrule
\end{tabularx}

\subsection{\big[viewing\big]}
%\label{sec:config.ini:viewing}

This section defines which objects, labels, lines, grids etc.\ you
want to see on startup. Set those to \emph{true}. Most items can be
toggled with hotkeys or switched in the GUI.

\begin{longtable}{l|l|p{77mm}}
\toprule
\emph{ID} & \emph{Type} & \emph{Description}\\\midrule
flag\_asterism\_drawing         & bool  & Display asterism line drawing\\%\midrule
flag\_asterism\_name            & bool  & Display asterism names\\%\midrule
flag\_rayhelper\_drawing        & bool  & Display the ray helper line drawing\\%\midrule
flag\_constellation\_drawing    & bool  & Display constellation line drawing\\%\midrule
flag\_constellation\_name       & bool  & Display constellation names\\%\midrule
flag\_constellation\_art        & bool  & Display constellation art\\%\midrule
flag\_constellation\_boundaries & bool  & Display constellation boundaries \\%\midrule
flag\_constellation\_isolate\_selected  & bool & If \emph{true}, constellation lines, boundaries and art will be limited to the constellation of the selected star, 
                                                 if that star is ``on'' one of the constellation lines.\\%\midrule
flag\_constellation\_pick     & bool & Set to \emph{true} if you only want to see the line drawing, art and name of the selected constellation star\\\midrule
%
flag\_isolated\_trails        & bool & Set to \emph{true} if you only want to see the trail line drawn for the selected planet (asteroid, comet, moon)\\%\midrule
flag\_isolated\_orbits        & bool & Set to \emph{true} if you want to see orbits only for selected planet and their moons.\\%\midrule
flag\_azimutal\_grid          & bool & Display azimuthal grid \\%\midrule
flag\_equatorial\_grid        & bool & Display equatorial grid (of date) \\%\midrule
flag\_equatorial\_J2000\_grid & bool & Display equatorial grid (J2000) \\%\midrule
flag\_ecliptic\_grid          & bool & Display ecliptic grid (of date) \\%\midrule
flag\_ecliptic\_J2000\_grid   & bool & Display ecliptic grid (J2000) \\%\midrule
flag\_galactic\_grid          & bool & Display galactic grid (System~II)\\%\midrule
flag\_galactic\_equator\_line & bool & Display galactic equator line \\%\midrule
flag\_equator\_line           & bool & Display celestial equator line (of date) \\%\midrule
flag\_equator\_J2000\_line    & bool & Display celestial equator line (J2000) \\%\midrule
flag\_ecliptic\_line          & bool & Display ecliptic line (of date) \\%\midrule
flag\_ecliptic\_J2000\_line   & bool & Display ecliptic line (J2000) \\%\midrule
flag\_meridian\_line          & bool & Display meridian line \\%\midrule
flag\_prime\_vertical\_line   & bool & Display Prime Vertical line (East-Zenith-West) \\%\midrule
flag\_colure\_lines           & bool & Display colure lines (Celestial Pole-\Aries/\Cancer/\Libra/\Capricorn) \\%\midrule
flag\_cardinal\_points        & bool & Display cardinal points\\\midrule
cardinal\_font\_size          & int  & Font size for cardinal points. Typical value: \emph{24}\\%\midrule
subcardinal\_font\_size       & int  & Font size for subcardinal points. Typical value: \emph{18}\\%\midrule
% TODO: This should go into the projection group:
flag\_gravity\_labels         & bool & Set to \emph{true} if you want labels to undergo gravity (top side of text points toward zenith). Useful with dome projection.\\\midrule
flag\_moon\_scaled            & bool & Keep it \emph{false} if you want to see the real moon size \\%\midrule
moon\_scale                   & float & Sets the moon scale factor, sometimes useful to correlate to our perception of the moon's size. Typical value: \emph{4}\\

flag\_minorbodies\_scaled     & bool & Set to \emph{true} if you want to see minor bodies enlarged \\
minorbodies\_scale            & float & Set scale factor for minor bodies, to see them in unnatural detail. Typical value: \emph{10}\\
\midrule
constellation\_art\_intensity      & float & brightness [0\ldots1] of the constellation art images. Typical value: \emph{0.5}\\%\midrule
constellation\_art\_fade\_duration & float & time the constellation art takes to fade in or out, in seconds. Typical value: \emph{1.5}\\%\midrule
constellation\_font\_size          & int   & font size for constellation labels\\%\midrule
constellation\_line\_thickness     & int   & thickness of constellation lines [1\ldots5]. Typical value: \emph{1}\\%\midrule
asterism\_font\_size               & int   & font size for asterism labels\\%\midrule
asterism\_line\_thickness          & int   & thickness of the asterism lines [1\ldots5]. Typical value: \emph{1}\\%\midrule
rayhelper\_line\_thickness         & int   & thickness of the ray helper lines [1\ldots5]. Typical value: \emph{1}\\\midrule
flag\_night                        & bool  & Enable night mode (red-only mode) on startup\\\midrule
light\_pollution\_luminance        & float & Sets the level of the light pollution simulation\\\midrule %% TODO: WHAT IS THIS?
use\_luminance\_adaptation         & bool  & Enable dynamic eye adaptation\\\midrule %% TODO: DESCRIBE!
sky\_brightness\_label\_threshold  & float & Sky brightness [$cd/m^2$] to show infotext in black \\ %% 
\bottomrule
\end{longtable}



\subsection{\big[DialogPositions\big]}
%\label{sec:config.ini:DialogPositions}

By default, GUI panels appear centered in the screen. You can move them to your favorite location, 
and on next start they will appear on that location instead.
The entries in this section define the upper left pixel coordinate where panels are stored.

Examples:

\noindent%
\begin{tabularx}{\textwidth}{l|l|l|X}
\toprule
\emph{ID}     & \emph{Type} & \emph{Example}&\emph{Description}     \\\midrule
Help          & int,int     &  58,39        & Position of Help panel\\%\midrule
DateTime      & int,int     &  1338,944     & Position of time panel\\\bottomrule
\end{tabularx}

%\newpage

\subsection{\big[DialogSizes\big]}
%\label{sec:config.ini:DialogSizes}

\newFeature{v0.15.1}
GUI panels can be resized by dragging on their lower corners or borders.  
Enlarged sizes are stored here, and on next start they will appear in this size.
The entries in this section define the size.

Examples:

\noindent%
\begin{tabularx}{\textwidth}{l|l|l|X}
\toprule
\emph{ID}     & \emph{Type} & \emph{Example}&\emph{Description}      \\\midrule
Help          & int,int     &  895,497      & Size of Help panel     \\%\midrule
AstroCalc     & int,int     &  913,493      & Size of AstroCalc panel\\\bottomrule
\end{tabularx}


\subsection{\big[hips\big]}
\label{sec:config.ini:hips}

This section defines the source list of the HiPS surveys, as well as other
HiPS related options.
\newFeature{v0.18.0}

\noindent%
\begin{tabularx}{\textwidth}{l|l|X}\toprule
\emph{ID}     & \emph{Type} & \emph{Description}\\\midrule
show          & bool        & show HiPS at startup? \\
sources/size  & int         & Number of HiPS survey sources\\%\midrule
sources/n/url & string      & uri of the $n$th source hipslist file ($n \in [1 \ldots \mathtt{sources/size}]$)\\
visible/url   & bool        & HiPS configured to be displayed. (Stores current setting on program exit)\\\bottomrule
\end{tabularx}


%%% Local Variables: 
%%% mode: latex
%%% TeX-PDF-mode: t
%%% TeX-master: "guide"
%%% End: 

